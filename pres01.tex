\documentclass{beamer}
\usetheme{Boadilla}
\usepackage[utf8]{inputenc}

\usepackage{color}
\usepackage{listings}

\lstdefinestyle{BashInputStyle}{
  language=bash,
  basicstyle=\small\sffamily,
  numbers=none,
  %numberstyle=\tiny,
  %numbersep=3pt,
  frame=tb,
  columns=fullflexible,
  backgroundcolor=\color{yellow!20},
  linewidth=0.9\linewidth,
  xleftmargin=0.1\linewidth
}


\title{Python-Workshop}
\subtitle{für blutige Anfänger*innen}
\author{min, ente}
\institute{Metalab Institute of Technology}
\date{\today}

\begin{document}

\begin{frame}
\titlepage
\end{frame}


% TOC, not sure we need it
%
%\section{Section 1}
%\subsection{sub a}
%
%\begin{frame}
%\frametitle{Outline}
%\tableofcontents
%\end{frame}

\begin{frame}
\frametitle{Inhalte des Kurses}
\begin{itemize}
	\item Python-Grundlagen \& Install Workshop
	\item Informatik-Grundlagen
	\item Datenstrukturen
	\item Logik Basics
	\item Kontrollstrukturen
	\item Funktionen
	\item Datentypen (int, float, str, list)
	\item Exception Handling
	\item Klassen, Objekte, Methoden, Vererbung
	\item Funktionale Programmierung
	\item Libraries
	% Die hatten wir vergessen oder außer Acht gelassen, wäre auf jeden Fall eher später als früher dran
	%\item Generatoren
	%\item Dekoratoren
\end{itemize}
\end{frame}

\begin{frame}
\frametitle{Inhalte dieser Einheit}
\begin{itemize}
	\item		tbd
\end{itemize}
\end{frame}

\begin{frame}
\frametitle {Warum Python?}
\begin{itemize}
	\item Einsteigerfreundlich
	\item Eingebaute Dokumentation
	\item Interaktiver Modus (Kommandos direkt ausprobieren)

	\item "Batteries Included"
	\begin{itemize}
		\item Python kommt mit einem großen Funktionsumfang in seiner Standard-Library
		\item z.B. File IO, random number generator,
			Sortieralgorithmen, Netzwerkprotokolle,
			Datenstrukturen, Syscall-Interface, ...
		\item Paketmanager zum Installieren von Extensions
	\end{itemize}

	\item Populär
	\begin{itemize}
		\item viel Support, große Community, viele fertige 3rd-Party-Libraries
		\item viele große Pakete für Webentwicklung
		\item Code ist durch die Größe der Community im Normalfall gut getestet
	\end{itemize}

\end{itemize}
\end{frame}

\begin{frame}
\frametitle {Warum Python?}
\begin{itemize}
	
	\item Wird von vielen bekannten Diensten verwendet 
	\begin{itemize}
		\item z.B. Youtube, Dropbox, Reddit, Instagram, das MOS, ...
	\end{itemize}

	\item Open Source 
	\begin{itemize}
		\item der Code des Python-Interpreters selbst ist offen und kann von allen gelesen, verstanden und verändert werden
	\end{itemize}

	\item Plattformunabhängig 
	\begin{itemize}
		\item der gleiche Code läuft auf allen Betriebsystemen, mit Ausnahme von plattformspezifischem Code wie z.B. bestimmte grafische Oberflächen oder das Syscall-Interface - später mehr
	\end{itemize}
	
	\item Wenig weirde Sprachfeatures und unerwartetes Verhalten
	\begin{itemize}
		\item Ist natürlich subjektiv, aber weniger merkwürdige Syntax-Quirks als z.B. PHP oder Perl.
		\item Es gibt einen offiziellen Style Guide (PEP-8) mit Best Practice
		\item Dadurch sieht der Code von unterschiedlichen Leuten auch etwas einheitlicher aus.
	\end{itemize}

\end{itemize}
\end{frame}

\begin{frame}
\frametitle {Ausführungsmodell}
Es gibt zwei Modi zur Ausführung von Python-Code:
\begin{itemize}
	\item Interaktiver Modus
	\begin{itemize}
		\item Einzelne Befehle ausführen
		\item Ergebnis des Befehls sofort sichtbar
		\item zum experimentieren
		\item als Taschenrechner
	\end{itemize}

	\item Scriptmodus
	\begin{itemize}
		\item Programm wird in Datei(en) geschrieben
		\item Ganzes Programm wird in einem abgearbeitet
		\item für Projekte
	\end{itemize}
\end{itemize}
\end{frame}

\begin{frame}
\frametitle {Installation von Python}

\begin{itemize}
	\item Wir verwenden in diesem Kurs IDLE
	\item Die Installation ist Betriebsystemabhängig
	\item ... aber grundsätzlich sehr einfach

	\begin{alertblock}{Achtung}
		Wir verwenden Python 3! (nicht 2)
	\end{alertblock}
\end{itemize}
\end{frame}

\begin{frame}
\frametitle {GNU/Linux}
	%\begin{lstlisting}
	bleh
	%\end{lstlisting}
	%\lstinputlisting[caption=Debian/Mint/Ubuntu, style=BashInputStyle]{hello.c}
\end{frame}


\end{document}
